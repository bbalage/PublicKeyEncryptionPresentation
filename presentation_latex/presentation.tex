\documentclass{beamer}
\usepackage[hungarian]{babel}
\usepackage[utf8]{inputenc}
\usepackage{amsmath}

\usetheme{Boadilla}

\title{Nyilvános kulcsú titkosítás}
\author{Bolyki Balázs}
\institute{Miskolci Egyetem}
\date{\today}

\begin{document}

\titlepage

\begin{frame}
    \frametitle{Alapgondolat}
    \textbf{Kérdés}: Hogyan tud két fél titkosítva információt cserélni egymással úgy, hogy soha nem találkoztak?

    \textbf{Válasz}: Nyilvános kulcsú (vagy asszimetrikus) titkosítással.

    \textbf{Két kulcs van}: Egy publikus és egy privát.

    \textbf{Postás példa}:

    \begin{itemize}
        \item Laci szeretne elküldeni egy dobozt Karcsinak.
        \item A dobozt csak Karcsi tudhatja kinyitni.
        \item Ezért Karcsi a postán hagy egy lakatot, amihez rendelkezik kulccsal.
        \item Laci a postára megy, felrakja a dobozra Karcsi lakatját, és elküldi postán a dobozt.
        \item Karcsi a kulcsával leszedi a lakatot, és kinyitja a dobozt.
    \end{itemize}

    A példában a lakat a nyilvános kulcs, a lakatkulcs pedig a titkos kulcs.

\end{frame}

\begin{frame}
    \frametitle{Működés általános leírása}

    \textbf{Kulcsok}:
    \begin{itemize}
        \item \textbf{Publikus kulcs}: Mindenki számára elérhető.
        \item \textbf{Privát kulcs}: Csak az adott fél számára elérhető.
    \end{itemize}
    \textbf{Megkötések}:
    \begin{itemize}
        \item A publikus kulccsal titkosított üzenet csak a privát kulccsal dekódolható.
        \item A privát kulccsal titkosított üzenet csak a publikus kulccsal dekódolható.
    \end{itemize}

    \textbf{Hol használják}:
    \begin{itemize}
        \item TLS/SSL (certificate authority)
        \item Digitális aláírás
        \item Blockchain (kriptovaluta)
    \end{itemize}
\end{frame}

\begin{frame}
    \frametitle{Formális meghatározás}

    Jelölje $s$ a titkos kulcsot, $p$ a publikus kulcsot, $M$ az üzenetet, $f$ pedig a titkosítás műveletét.

    Ekkor igaz az, hogy:
    \begin{align*}
        M' & = f(M, p)  \\
        M  & = f(M', s)
    \end{align*}
\end{frame}

\begin{frame}
    \frametitle{RSA}

    Text (TODO)
\end{frame}

\end{document}